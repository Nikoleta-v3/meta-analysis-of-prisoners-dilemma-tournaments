\documentclass{article}
\usepackage[margin=2.5cm, includefoot, footskip=30pt]{geometry}

\setlength{\parindent}{0em}
\setlength{\parskip}{1em}
\renewcommand{\baselinestretch}{1}

%%%%Packages%%%%
\usepackage{amsmath}
\usepackage{booktabs}
\usepackage{graphics}
\usepackage{multicol}
\usepackage[ruled,vlined]{algorithm2e}
\usepackage{setspace}
\usepackage{graphicx}
\usepackage{subcaption}
\usepackage{hyperref}
\usepackage{color,colortbl}
\usepackage{array}
\usepackage{booktabs}
\usepackage{tabularx}
\usepackage{wrapfig, blindtext}
\usepackage{soul}
\usepackage[table]{xcolor}
%%%%%%%%%%%%%%%%%

\definecolor{Gray}{gray}{0.92}
\usepackage[first=0,last=9]{lcg}
\newcommand{\ra}{\rand0.\arabic{rand}}

\newcommand{\uniquenumberofseeds}{\input{unique_number_of_seeds.tex}}
\newcommand{\numberofalltournaments}{\input{number_of_all_tournaments.tex}}
\newcommand{\numberofstrategies}{\input{unique_number_of_strategies.tex}}
\def\axelrod{\texttt{Axelrod-Python }}

\setlength{\tabcolsep}{3pt}

\title{Properties of Winning Iterated Prisoner's Dilemma Strategies.}
\author{Nikoleta E. Glynatsi, Vincent A. Knight, Marc Harper}
\date{}

\begin{document}

\maketitle

\begin{abstract}
Researchers have explored the performance of Iterated Prisoner's Dilemma strategies
for decades: from the celebrated performance of Tit for Tat, to the
introduction of the zero-determinant strategies, to the use of sophisticated learning
structures such as neural networks, many new strategies have been introduced and tested
in a variety of tournaments and population dynamics. Typical results in the literature,
however, rely on performance against a small number of somewhat arbitrarily selected
strategies in a small number of tournaments, casting doubt on the generalizability
of conclusions. We analyze a large collection of \numberofstrategies
strategies in \numberofalltournaments tournaments, present the top performing strategies across multiple
tournament types, and distill their salient features.
The results show that there is not yet a single
strategy that performs well in diverse Iterated Prisoner's Dilemma scenarios,
nevertheless there are several properties that heavily influence the best performing
strategies. This refines the properties described by R. Axelrod in light of
recent and more diverse opponent populations to: be nice, be provocable and generous,
be a little envious, be clever, and adapt to the environment. More precisely,
we find that strategies perform best when their probability of cooperation
matches the total tournament population's aggregate cooperation probabilities,
or a proportion thereof in the case of noisy and probabilistically ending tournaments,
and that the manner in which a strategy achieves the ideal cooperation rate is crucial.
The features of high performing strategies help cast some light on why strategies such as Tit For Tat
performed historically well in tournaments and why zero-determinant strategies
typically do not fare well in tournament settings.
\end{abstract}

\section{Background}

The Iterated Prisoner's Dilemma (IPD) is a repeated two player game that models
behavioural interactions, specifically interactions where
self-interest clashes with collective interest. At each turn of the game both
players, simultaneously and independently, decide between cooperation (\(C\)) and
defection (\(D\)), given memory of all prior interactions. The payoffs for each
player, at each turn, is influenced by their own choice and the choice of the
other player. The payoffs of the game are defined by:

\begin{center}
{\renewcommand{\arraystretch}{2}%
\begin{tabular}{c|c|c}
& Cooperate (\(C\)) & Defect (\(D\)) \\
\hline
Cooperate (\(C\)) & \(R\) & \(S\) \\
\hline
Defect (\(D\)) & \(T\) & \(P\) \\
\end{tabular}}
\end{center}

where typically \(T > R > P > S\) and \(2R > T + S\). The most common values used in
the literature~\cite{Axelrod1981} are $R=3, P=1, T=5, S=0$. These values are also
used in this work.

Conceptualising strategies and understanding the best way of playing the game
has been of interest to the scientific community since the formulation of the
game in 1950~\cite{Flood1958}. Following the computer tournaments of R. Axelrod in the
1980's~\cite{Axelrod1980a, Axelrod1980b}, a strategy's performance in a round
robin computer tournament became a common evaluation technique for newly designed
strategies. Many tournaments have followed Axelrod's~\cite{Beaufils1997, Bendor1991,
Harper2017, Kendall2007, Stephens2002, Stewart2012} and now the
literature and various codebases contain hundreds of strategies.

The winner of both of R. Axelrod's tournaments~\cite{Axelrod1980a, Axelrod1980b}
was the simple strategy Tit For Tat (TFT) which cooperates
on the first turn and thereafter copies the previous action of its opponent,
retailiating against defections with a defection, and forgiving a defection if followed
by a cooperation. R. Axelrod concluded that the strategy's robustness was due to four
properties, which he adapted into four suggestions on doing well in an IPD:

\begin{itemize}
    \item Do not be envious by striving for a payoff larger than the opponent's payoff
    \item Be ``nice''; Do not be the first to defect
    \item Reciprocate both cooperation and defection; Be provocable to retaliation and forgiveness
    \item Do not be too clever by scheming to exploit the opponent
\end{itemize}

Forgiveness is a strategy's ability to go from a \(DC\) to \(C\) aiming to achieve
mutual cooperation again, the only way TFT would end up in \(DC\), in
environments without noise, is if it had received a defection and then retaliated.
Subsequently, TFT would forgive an opponent that apologises (in a \(DC\) round)
by returning to cooperation, since mutual cooperation is better than mutual defection.
As a result of the strategy's strong performance in both tournaments, and moreover in a
series of evolutionary experiments~\cite{Axelrod1981}, TFT was often
claimed to be a highly robust (and sometimes the most robust) strategy for the IPD.

There are strategies which have built upon TFT and the reciprocity based approach.
In~\cite{Beaufils1997} a strategy called Gradual was introduced, constructed to have the
same qualities as those of TFT with one addition. Gradual has a memory
of the previous rounds of play of the game, recording the number of defections
by the opponent and punishing them with a growing number of defections. It
then enters a calming state in which it cooperates for two rounds. A
strategy with the same intuition as Gradual is Adaptive Tit for
Tat~\cite{tzafestas-2000a}. Adaptive Tit for Tat maintains a continually updated estimate of the
opponent's behaviour and uses this estimate to condition its future actions.

Other work has built upon the limitations of TFT, and in some cases have shown
that suggestions made by R. Axelrod did not necessarily apply in alternative environmental settings.
In~\cite{Bendor1991, Donninger1986, Molander1985, Hammerstein1984} it was shown
that TFT suffered in environments with noise. This was
mainly due to the strategy being too provocable and its lack of generosity and contrition. Since TFT immediately
punishes a defection, in a noisy environment it can get stuck in a
repeated cycle of defections and cooperations. Some new strategies, more
robust in tournaments with noise, were soon introduced, including
Nice and Forgiving~\cite{Bendor1991}, Generous Tit For Tat~\cite{Nowak1992},
and Pavlov (aka Win Stay Lose Shift)~\cite{Nowak1993}, as well as later
variants such as OmegaTFT \cite{kendall2007iterated}. The introduction of a new strategy
is often accompanied by a claim that the new strategy is the best known, despite
only being tested against a small number of opponents or against specific classes of
opponents not necessarily representative of all possible or all published strategies.
The lack of testing against formally defined strategies and tournament winners
is understandable given the effort required to implement the hundreds of published IPD strategies.
Implementing prior strategies faithfully is often extremely difficult or
impossible due to insufficient descriptions and lack of published
implementations or code.
This calls into question any claims of superiority or robustness of newly introduced strategies.

A set of envious IPD strategies were introduced called zero-determinant strategies (ZDs)
in~\cite{Press2012}. These strategies attempt to force a linear relationship between
stationary payoffs against other memory-one opponents, potentially ensuring that they
receive a higher average payout. While ZDs were introduced with a small tournament in
which some were reportedly successful \cite{Stewart2012}, this result has not generally
held in future work. In~\cite{Harper2017} a series of strategies trained using
reinforcement learning were introduced, and a tournament containing over 200 strategies
featured no ZDs ranking in top spots. Instead, the top ranked strategies
were a set of ``clever'' (in the sense of R. Axelrod's characteristics) trained
strategies based on lookup tables~\cite{Axelrod1987}, hidden Markov
models~\cite{Harper2017}, and finite state automata~\cite{Miller1996}.
Similarly, in \cite{mathieu2017}, a set of evolutionarily-trained strategies,
and a pre-selected set of known strategies, outperformed a collection of 12 ZDs.

Though only select pieces of work have been discussed, there is a broad collection
of strategies in the literature, and new strategies and competitions are
published frequently~\cite{Glynatsi2019}. The question, however, still remains
the same: what is the best way to play the game?

Compared to other works, where typically a few selected or introduced strategies
are evaluated on a small number of tournaments and/or small number of opponents,
this manuscript evaluates the performance of \numberofstrategies
strategies in \numberofalltournaments tournaments. Furthermore, a large portion
of the strategies used in this manuscript are drawn from known and named strategies
in IPD literature, including many previous tournament winners,
in contrast to other work that may have randomly generated many essentially arbitrary
strategies (typically restrained to a class such as memory-one strategies,
or those of a certain structural form such as finite state machines or deterministic
memory-two strategies). Additionally, our tournaments come in a
number of variations including standard tournaments emulating R. Axelrod's original tournaments,
tournaments with noise, probabilistic match length, and both noise and probabilistic match length.
This diversity of strategies and tournament types yields new insights and tests
earlier claims in alternative settings against known powerful strategies.

The later part of the paper evaluates the impact of various strategy features
on the performance of the strategies using standard statistical and machine learning techniques. These
features include measures regarding a strategy's behaviour and measures regarding
the tournaments. This rigorous analysis reinforces the discussion started by R. Axelrod
and concludes that the properties of a successful strategy in the IPD are:

\begin{itemize}
    \item \st{Do not be envious} Be a little bit envious
    \item Be ``nice'' in non-noisy environments or when game lengths are longer
    \item Reciprocate both cooperation and defection appropriately;
    Be provocable in tournaments with short matches, and generous when matches are longer
    \item \st{Do not be too clever} It's ok to be clever
    \item Adapt to the environment; Adjust to the mean population cooperation
\end{itemize}

The different tournament types as well as the data collection, made
possible due to an open source library called Axelrod-Python (APL),
are covered in Section~\ref{section:data_collection}. The raw and processed data
sets have been made publicly available~\cite{data, raw_data} and can be used
for further analysis and insights.
Section~\ref{section:top_performances}, focuses on the best performing
strategies for each type of tournament and overall.
Section~\ref{section:evaluation_of_performance}, explores the traits which
contribute to good performance, and finally the results are summarised in
Section~\ref{section:conclusion}. This manuscripts uses several parameters,
introduced in the following sections. The full set of
parameters and their definitions are given in Appendix~\ref{app:parameters}.

\section{Data collection}\label{section:data_collection}

The data set generated for this manuscript was created using \axelrod version 3.0.0.
\axelrod enables the simulation of Iterated Prisoner's Dilemma
computer tournaments and contains an extensive list of strategies. Most of these
strategies are described in the literature, with a few exceptions contributed
specifically to the package. In this paper, we use a total of \numberofstrategies
strategies. You can find a list of these strategies in the Supplementary Material.

The package supports several tournament types, and this work considers standard,
noisy, probabilistic ending, and noisy probabilistic ending tournaments.
{\it Standard tournaments} are similar to Axelrod's well-known
tournaments~\cite{Axelrod1980a}. In these tournaments, there are \(N\)
strategies, and each strategy plays an iterated game with \(n\) turns against
all other strategies, not including self-interactions.
{\it Noisy tournaments} also involve \(N\) strategies and \(n\) turns, but
in each turn, there is a probability \(p_n\) that a player's action is
flipped.
{\it Probabilistic ending tournaments} consist of \(N\) strategies, and
after each turn, a match between strategies ends with a given probability
\(p_e\).
Finally, {\it noisy probabilistic ending tournaments} incorporate both a
noise probability \(p_n\) and an ending probability \(p_e\). For smoother
results, each tournament is repeated \(k\) times, and this repetition factor was
allowed to vary to assess the impact of smoothing. The winner of each tournament
is determined based on the average score achieved by a strategy from the entire
set of repetitions, not by the number of wins.
The process of collecting tournament results is outlined in
Algorithm~\ref{algorithm:data_generation}. For each trial, we choose a random
size \(N\) is selected, and a random list of \(N\) strategies from the
\numberofstrategies available. Subsequently, standard, noisy, probabilistic
ending, and noisy probabilistic ending tournaments are conducted for the
selected list of strategies. The parameters for the tournaments, as well as the
number of repetitions, are chosen once for each trial.


\begin{algorithm}[!htbp]
    \setstretch{1.35}
    \For{\text{seed} $\in [0, 11420]$}{
        $N \gets \text{randomly select integer}\in [3, 195]$\;
        $\text{players} \gets  \text{randomly select $N$ players}$\;
        $k \gets  \text{randomly select integer}\in [10, 100]$\;
        $n \gets  \text{randomly select integer}\in [1, 200]$\;
        $p_n \gets  \text{randomly select float}\in [0, 1]$\;
        $p_e \gets   \text{randomly select float}\in [0, 1]$\;
        \vspace{0.4cm}
        $\text{result standard}$ $\gets$ Axelrod.tournament$(\text{players}, n, k)$\;
        $\text{result noisy}$ $\gets$ Axelrod.tournament$(\text{players}, n, p_n, k)$\;
        $\text{result probabilistic ending}$ $\gets$ Axelrod.tournament$(\text{players}, p_e, k)$\;
        $\text{result noisy probabilistic ending}$ $\gets$ Axelrod.tournament$(\text{players}, p_n, p_e, k)$\;

    }
    \KwRet{result standard, result noisy, result probabilistic ending,
    result noisy probabilistic ending}\;
    \caption{Tournament Data Collection Algorithm}
    \label{algorithm:data_generation}
\end{algorithm}

We have run a total of \uniquenumberofseeds trials of
Algorithm~\ref{algorithm:data_generation}. For each trial, we collect the
results for four different tournaments, resulting in a total of
\numberofalltournaments $(\uniquenumberofseeds \times 4)$ tournament results.
Each tournament outputs a result summary in the form of
Table~\ref{table:output_result}. Each strategy has participated, on average, in
5154 tournaments of each type. The strategy with the maximum participation in
each tournament type is Inverse Punisher with 5639 entries. The strategy with
the minimum entries is EvolvedLookerUp 1 1 1, which was selected in 4693 trials.
During the data collection process, we allowed the probabilities of noise and
tournament ending to vary between 0 and 1. However, commonly used values for
these probabilities are \(p_n < 0.5\) and \(p_e < 0.1\) (for more details, see the
Supplementary Material). Therefore, in this context, we will focus on these
subsets for the three tournament types: noisy, probabilistic ending, and noisy
probabilistic ending. The results presented here pertain to these subsets.
Specifically, there are a total of 5650, 1134, and 565 runs for each of the
three tournament types, respectively. We also provide an analysis of the paper
considering the entire datasets, and these results are presented in the
Supplementary Material.

\newcolumntype{g}{>{\columncolor{Gray}}c}
\begin{table}[!htbp]
    \begin{center}
    \resizebox{\textwidth}{!}{
    \begin{tabular}{ccccccgcgcgcgcg}
    \toprule
    & & & & & &   \multicolumn{8}{g}{Rates}  \\
    Rank & Name & Median score & Cooperation rating $(C_r)$ & Win & Initial C &
    CC & CD & DC & DD & CC to C & CD to C & DC to C & DD to C \\
    0 &  EvolvedLookerUp2 2 2 & 2.97 & 0.705 & 28.0 & 1.0 & 0.639 & 0.066 & 0.189 &
    0.106 & 0.836 & 0.481 & 0.568 & 0.8 \\
    1 &  Evolved FSM 16 Noise 05 & 2.875 & 0.697 & 21.0 & 1.0 & 0.676 &
    0.020 & 0.135 & 0.168 & 0.985 & 0.571 & 0.392 & 0.07 \\
    2 & PSO Gambler 1 1 1 & 2.874 & 0.684 &  23.0 &     1.0 &    0.651 &    0.034 &    0.152 &    0.164
    & 1.000 & 0.283 & 0.000 & 0.136 \\
    3 &  PSO Gambler Mem1 &  2.861 &        0.706 &  23.0 &      1.0 &    0.663
    &    0.042 &    0.145 &    0.150 &  1.000 &  0.510 &  0.000 &  0.122 \\
    4 &          Winner12 &  2.835 &        0.682 &  20.0 &      1.0 &
    0.651 &    0.031 &    0.141 &    0.177 &  1.000 &  0.441 &  0.000 &  0.462 \\
    $\dots$ & $\dots$ & $\dots$ & $\dots$ & $\dots$ & $\dots$ & $\dots$ & $\dots$ &
    $\dots$ & $\dots$ & $\dots$ & $\dots$ & $\dots$ & $\dots$ \\
    \bottomrule
    \end{tabular}}
\end{center}
\caption{\textbf{Result Summary Example of a Tournament.}
A result summary consists of \(N\) rows, with each row containing information
for each strategy that participated in the tournament. This information includes
the strategy's rank (\(R\)), median score, the cooperation rate (\(C_r\)), the number of
match wins, and the probability that the strategy cooperated in the opening
move. Additionally, it provides the probabilities of a strategy being in any of
the four states ($CC, CD, DC, DD$) and the cooperation rate after each state.}\label{table:output_result}
\end{table}

\section{Top ranked strategies}\label{section:top_performances}

A strategy has participated in multiple tournaments of each type, and to
evaluate its overall performance, we introduce a measure called the {\textit
normalized rank}. In each tournament, the strategies receive a rank, where 0
denotes that the strategy was the winner, and \(N-1\) indicates that the
strategy came last in the tournament. The normalized rank, denoted as \(r\), is
calculated as \(r = \frac{R}{N-1}\). Thus, the rank a strategy achieved over the
number of players in the tournament. The performance of the strategies is
assessed based on the {\it median of the normalized rank}, denoted as
\(\bar{r}\).

For example, let's consider the well-known strategies Tit For Tat and
Gradual. Each strategy participated in several tournaments of each type (see
Figure~\ref{fig:normalised_rank_distributions}).
We show the distribution of the ranks of these
strategies in each of the four tournaments. We can observe that in tournaments
with the presence of noise, Tit For Tat has a normally distributed normalized
rank around 1/2. In tournaments without noise, the strategy performs better,
achieving its best performance in probabilistic ending tournaments with a median
normalized rank of 0.298.
In comparison, Gradual's performance has longer tails, indicating that there
were tournaments where the strategy performed very well or very poorly. Overall,
Gradual achieves a lower median rank, signifying that it performs better than
Tit For Tat except in the case of noisy and probabilistic ending tournaments.

\begin{figure}[h]
    \centering
    \includegraphics[width=\textwidth]{../images/normalised_rank_distributions.pdf}
    \caption{\textbf{Examples of normalized rank distributions for two
    strategies, Tit For Tat and Gradual.} We plot the distributions of \(r\) for
    the two strategies in the four tournament types. As a reminder, lower values
    of \(r\) correspond to better performances. In each plot, we also show the
    number of data points. Both strategies participated in a similar number of
    tournaments. Based on the median rank, which we use in this work to define
    overall performance, Tit For Tat performs best in probabilistic ending
    tournaments, whereas Gradual was in standard tournaments. The best
    performance of the Gradual strategy has been in standard tournaments where
    it achieved a \(\bar{r}\) of 0.34.}
    \label{fig:normalised_rank_distributions}
\end{figure}


\begin{figure}[t]
    \centering
    \includegraphics[width=.8\textwidth]{../images/top_performers.pdf}
    \caption{\(r\) distributions of the top 15 strategies in different
    environments. A lower value of \(\bar{r}\) corresponds to a more successful
    performance. A strategy's \(r\) distribution skewed towards zero indicates
    that the strategy ranked highly in most tournaments it participated in. Most
    distributions are skewed towards zero except the distributions with
    unrestricted noise, supporting the conclusions from
    Table~\ref{table:top_performances}.}\label{fig:r_distributions}
\end{figure}

\newcolumntype{g}{>{\columncolor{Gray}}l}
\begin{table}[t]
    \begin{center}
    \resizebox{\textwidth}{!}{
        
% \midrule
% & Name & $\bar{r}$ &                 Name & $\bar{r}$ &               Name & $\bar{r}$ &                 Name & $\bar{r}$ &                 Name & $\bar{r}$ &                 Name & $\bar{r}$ \\
% \midrule
% 0  &            Evolved HMM 5 &     0.007 &                 Grumpy &     0.140 &                         DBS &     0.000 &          Fortress4 &     0.013 &             Evolved FSM 16 &     0.000 &           Alternator &     0.304 \\
% 1  &           Evolved FSM 16 &     0.010 &                    $e$ &     0.194 &     Evolved FSM 16 Noise 05 &     0.008 &           Defector &     0.014 &    Evolved FSM 16 Noise 05 &     0.013 &              $\phi$  &     0.310 \\
% 2  &     EvolvedLookerUp2 2 2 &     0.011 &         Tit For 2 Tats &     0.206 &      Evolved ANN 5 Noise 05 &     0.013 &  Better and Better &     0.016 &                       MEM2 &     0.027 &                $e$   &     0.312 \\
% 3  &  Evolved FSM 16 Noise 05 &     0.017 &  Slow Tit For Two Tats &     0.210 &                 BackStabber &     0.024 &    Tricky Defector &     0.019 &              Evolved HMM 5 &     0.044 &               $\pi$  &     0.317 \\
% 4  &        PSO Gambler 2 2 2 &     0.021 &           Cycle Hunter &     0.215 &               DoubleCrosser &     0.025 &          Fortress3 &     0.022 &       EvolvedLookerUp2 2 2 &     0.049 &    Limited Retaliate &     0.353 \\
% 5  &              Evolved ANN &     0.029 &         Risky QLearner &     0.222 &               Evolved ANN 5 &     0.028 &     Gradual Killer &     0.025 &       Spiteful Tit For Tat &     0.060 &     Anti Tit For Tat &     0.354 \\
% 6  &            Evolved ANN 5 &     0.034 &            Retaliate 3 &     0.229 &                 Evolved ANN &     0.038 &         Aggravater &     0.028 &           Nice Meta Winner &     0.068 &  Limited Retaliate 3 &     0.356 \\
% 7  &        PSO Gambler 1 1 1 &     0.037 &          Cycler CCCCCD &     0.235 &        Spiteful Tit For Tat &     0.051 &             Raider &     0.031 &         NMWE Finite Memory &     0.069 &          Retaliate 3 &     0.356 \\
% 8  &            Evolved FSM 4 &     0.049 &            Retaliate 2 &     0.239 &               Evolved HMM 5 &     0.051 &         Cycler DDC &     0.045 &         NMWE Deterministic &     0.070 &            Retaliate &     0.357 \\
% 9  &         PSO Gambler Mem1 &     0.050 &        Defector Hunter &     0.240 &              Level Punisher &     0.052 &        Hard Prober &     0.051 &                    Grudger &     0.070 &          Retaliate 2 &     0.358 \\
% 10 &                 Winner12 &     0.060 &              Retaliate &     0.242 &                   Omega TFT &     0.059 &         SolutionB1 &     0.060 &           NMWE Long Memory &     0.074 &  Limited Retaliate 2 &     0.361 \\
% 11 &             Fool Me Once &     0.061 &    Hard Tit For 2 Tats &     0.250 &                Fool Me Once &     0.059 &      Meta Minority &     0.061 &  Nice Meta Winner Ensemble &     0.076 &             Hopeless &     0.368 \\
% 12 &                      DBS &     0.071 &    Limited Retaliate 3 &     0.253 &  PSO Gambler 2 2 2 Noise 05 &     0.067 &              Bully &     0.061 &       EvolvedLookerUp1 1 1 &     0.077 &    Arrogant QLearner &     0.407 \\
% 13 &            DoubleCrosser &     0.072 &               ShortMem &     0.253 &              Evolved FSM 16 &     0.078 &             EasyGo &     0.071 &            NMWE Memory One &     0.080 &    Cautious QLearner &     0.409 \\
% 14 &              BackStabber &     0.075 &      Limited Retaliate &     0.257 &                  EugineNier &     0.080 &    Fool Me Forever &     0.071 &                   Winner12 &     0.085 &      Fool Me Forever &     0.418 \\
% \bottomrule
% \end{tabular}

\begin{tabular}{lggllggllggllr}
    \toprule
    & \multicolumn{2}{g}{Standard} & \multicolumn{2}{g}{Noisy (\(p_n < 0.5\))} & \multicolumn{2}{g}{Probabilistic ending (\(p_e < 0.1\))} &  \multicolumn{2}{c}{Noisy probabilistic ending} \\
    \toprule
    {} &                     Name & $\bar{r}$ &                       Name & $\bar{r}$ &                       Name & $\bar{r}$ &                       Name & $\bar{r}$ \\
    \midrule
    0  &            Evolved HMM 5 &     0.007 &                       MEM2 &     0.061 &             Evolved FSM 16 &       0.0 &                   Defector &     0.006 \\
    1  &           Evolved FSM 16 &      0.01 &       Spiteful Tit For Tat &     0.063 &    Evolved FSM 16 Noise 05 &     0.013 &          Better and Better &     0.011 \\
    2  &     EvolvedLookerUp2\_2\_2 &     0.011 &           Nice Meta Winner &     0.066 &                       MEM2 &     0.027 &                 Aggravater &     0.014 \\
    3  &  Evolved FSM 16 Noise 05 &     0.017 &                    Grudger &     0.067 &              Evolved HMM 5 &     0.043 &                  Fortress4 &     0.021 \\
    4  &        PSO Gambler 2\_2\_2 &     0.022 &    Meta Winner Long Memory &     0.073 &       EvolvedLookerUp2\_2\_2 &     0.049 &            Tricky Defector &     0.038 \\
    5  &              Evolved ANN &     0.029 &                   Forgiver &     0.074 &       Spiteful Tit For Tat &     0.059 &    Meta Winner Long Memory &     0.049 \\
    6  &            Evolved ANN 5 &     0.034 &               Fool Me Once &     0.074 &           Nice Meta Winner &     0.069 &     Meta Winner Memory One &      0.05 \\
    7  &        PSO Gambler 1\_1\_1 &     0.037 &                Meta Winner &     0.075 &         NMWE Finite Memory &     0.069 &  Meta Winner Finite Memory &      0.05 \\
    8  &            Evolved FSM 4 &     0.049 &     Meta Winner Memory One &     0.076 &         NMWE Deterministic &      0.07 &     Meta Winner Stochastic &     0.051 \\
    9  &         PSO Gambler Mem1 &      0.05 &  Meta Winner Finite Memory &     0.077 &                    Grudger &      0.07 &  Meta Winner Deterministic &     0.052 \\
    10 &                 Winner12 &      0.06 &  Meta Winner Deterministic &     0.078 &           NMWE Long Memory &     0.074 &                Meta Winner &     0.053 \\
    11 &             Fool Me Once &     0.061 &         NMWE Deterministic &     0.087 &  Nice Meta Winner Ensemble &     0.076 &       Meta Winner Ensemble &     0.059 \\
    12 &                      DBS &     0.071 &           NMWE Long Memory &     0.087 &       EvolvedLookerUp1\_1\_1 &     0.077 &                  Fortress3 &      0.07 \\
    13 &            DoubleCrosser &     0.072 &         CollectiveStrategy &     0.087 &            NMWE Memory One &      0.08 &         CollectiveStrategy &     0.077 \\
    14 &              BackStabber &     0.075 &                   Defector &     0.089 &            NMWE Stochastic &     0.085 &                   Prober 3 &      0.08 \\
    \bottomrule
    \end{tabular}
    
    }
\end{center}
\caption{Top performances for each tournament type based on $\bar{r}$. The
results of each type are based on 11420 unique tournaments. The
results for noisy tournaments with \(p_n < 0.1\) are based on 1151 tournaments,
and for probabilistic ending tournaments with \(p_e < 0.1\) on 1139. The top
ranks indicate that trained strategies perform well in a variety of
environments, but so do simple deterministic strategies. The normalised medians
are close to 0 for most environments, except environments with noise not
restricted to 0.1 regardless of the number of turns. Noisy and noisy probabilistic
ending tournaments have the highest medians.}
\label{table:top_performances}
\end{table}


The top 15 strategies for each tournament type, based on \(\bar{r}\), are
presented in Table~\ref{table:top_performances}, while the \(r\) distributions
for the top-ranked strategies can be found in Figure~\ref{fig:r_distributions}.
In standard tournaments, 10 out of the 15 top strategies were introduced
in~\cite{Harper2017}. These strategies are based on finite state automata (FSM),
hidden Markov models (HMM), artificial neural networks (ANN), lookup tables
(LookerUp), and stochastic lookup tables (Gambler). They have been trained using
reinforcement learning algorithms (evolutionary and particle swarm algorithms)
to perform well against a subset of the strategies in \axelrod in a standard
tournament. Thus, their performance in the specific setting was anticipated,
although still noteworthy given the random sampling of tournament participants.
DoubleCrosser and BackStabber, both from the \axelrod, use the number of turns and
are set to defect in the last two rounds. These strategies can be characterized
as “cheaters” because their source code allows them to know the number of turns
(unless the match has a probabilistic ending). These strategies were expected to
not perform as well in tournaments where the number of turns is not specified.
Finally, Winner 12~\cite{mathieu2017} and DBS~\cite{Au2006} are both from the
literature. DBS is a strategy specifically designed for noisy environments;
however, it ranks highly in standard tournaments as well. Similarly, the
fourth-ranked player, Evolved FSM 16 Noise 05, was trained for noisy tournaments
yet performs well in standard tournaments.

In the case of noisy tournaments, the
top-performing strategies include strategies specifically designed for noisy
tournaments. These are DBS, Evolved FSM 16 Noise 05, Evolved ANN 5 Noise 05, PSO
Gambler 2 2 2 Noise 05, and Omega Tit For Tat~\cite{kendall2007iterated}. Omega
TFT, a strategy designed to break the deadlocking cycles of \(CD\) and \(DC\)
that TFT can fall into in noisy environments, places 10th. The rest of the top
ranks are occupied by strategies that performed well in standard tournaments and
deterministic strategies such as Spiteful Tit For Tat~\cite{prison}, Level
Punisher~\cite{Eckhart2015}, Eugine Nier~\cite{lesswrong}.

The most effective strategies in probabilistic ending tournaments with \(p_e<
0.1\) are a series of ensemble Meta strategies, trained strategies which
performed well in standard tournaments, and Grudger~\cite{axelrodproject} and
Spiteful Tit for Tat~\cite{prison}. The Meta strategies~\cite{axelrodproject}
utilize a team of strategies and aggregate the potential actions of the team
members into a single action in various ways.

Overall, the analysis reveals that dominating strategies in standard tournaments
were those trained using reinforcement learning techniques. In standard
tournaments, these dominating strategies exhibited a clear trend of being
trained through reinforcement learning techniques. Additionally, in environments
with a noise probability strictly less than 0.1, successful strategies were
purposefully designed or trained to adapt to noisy conditions. Furthermore, in
tournaments with probabilistic endings, the highly ranked strategies leaned
towards defecting strategies and trained finite state automata, as demonstrated
by the works of Ashlock et al.~\cite{Ashlock2006,Ashlock2014}. Their prominence
in tournaments where the probability of the game ending after each turn exceeded
0.1 underscored their effectiveness. Similarly, in probabilistic tournaments
with \(p_e\) less than 0.1, the highly ranked strategies were characterized by
their reliance on the behavior of others. Notably, from the comprehensive set of
strategies considered, no strategy demonstrated consistent success in noisy
environments unless the noise value was limited to less than 0.1.

Though there is not a single strategy that consistently outranks all others in
any of the distinct tournament types or even across the tournament types, there
are specific types of strategies that have been repeatedly ranked in the top
ranks. These include strategies that have been trained, strategies that
retaliate, and strategies that adapt their behavior based on preassigned rules
to achieve the highest outcome. These results contradict some of Axelrod's
suggestions, and more specifically, the suggestions ``Do not be clever'' and
``Do not be envious''. We delve deeper into the crucial strategy features for
success in the following section.

\section{Evaluation of performance}\label{section:evaluation_of_performance}

For each strategy, for each tournament we have a variety of features, described in
Table~\ref{table:manual_features}. These features are measures regarding a
strategy's behaviour from the tournaments the strategies competed in as well as
intrinsic properties such as whether a strategy is deterministic or stochastic.

\newcolumntype{g}{>{\columncolor{Gray}}c}
\begin{table}[!htbp]
    \begin{center}
    \resizebox{.99\textwidth}{!}{
    \begin{tabular}{gcgcgc}
    \toprule
    feature & feature explanation &  source & value type & min value & max value \\
    \midrule
stochastic  &  If a strategy is stochastic & strategy classifier from APL & boolean  & Na &  Na \\
makes use of game &  If a strategy makes used of the game information & strategy classifier from APL & boolean  & Na &  Na \\
makes use of length &  If a strategy makes used of the number of turns & strategy classifier from APL & boolean  & Na &  Na \\
memory usage &  The memory size of a strategy divided by the number of turns & memory size from APL & float & 0 &  1 \\
SSE & A measure of how far a strategy is from ZD behaviour & method described in~\cite{Knight2019} & float & 0 & 1 \\
max cooperating rate $(C_{\text{max}})$  & The biggest cooperating rate in a given tournament  & result summary  & float & 0 & 1\\
min cooperating rate $(C_{\text{min}})$ & The smallest cooperating rate in a given tournament  & result summary  & float & 0 & 1\\
median cooperating rate $(C_{\text{median}})$ & The median cooperating rate in a given tournament  & result summary  & float & 0 & 1\\
mean cooperating rate $(C_{\text{mean}})$ & The mean cooperating rate in a given tournament  & result summary  & float & 0 & 1 \\
$C_r$ / $C_{\text{max}}$ & A strategy's cooperating rate divided by the maximum & result summary  & float & 0 & 1 \\
$C_{\text{min}}$ / $C_r$ & A strategy's cooperating rate divided by the minimum & result summary  & float & 0 & 1 \\
$C_r$ / $C_{\text{median}}$ & A strategy's cooperating rate divided by the median  & result summary  & float & 0 & 1\\
$C_r$ / $C_{\text{mean}}$ & A strategy's cooperating rate divided by the mean & result summary  & float & 0 & 1 \\
$C_r$ & The cooperating ratio of a strategy & result summary  & float & 0 & 1 \\
$CC$ to $C$ rate & The probability a strategy will cooperate after a mutual cooperation & result summary  & float & 0 & 1\\
$CD$ to $C$ rate & The probability a strategy will cooperate after being betrayed by the opponent & result summary  & float & 0 & 1 \\
$DC$ to $C$ rate & The probability a strategy will cooperate after betraying the opponent & result summary  & float & 0 & 1 \\
$DD$ to $C$ rate & The probability a strategy will cooperate after a mutual defection & result summary  & float & 0 & 1 \\
$p_n$ & The probability of a player's action being flip at each interaction & trial summary & float & 0 & 1 \\
$n$ & The number of turns & trial summary & integer & 1 & 200 \\
$p_e$ & The probability of a match ending in the next turn & trial summary & float & 0 & 1 \\
$N$ & The number of strategies in the tournament & trial summary & integer & 3 & 195 \\
$k$ & The number of repetitions of a given tournament & trial summary & integer & 10 & 100 \\
    \bottomrule
        \end{tabular}}
    \end{center}
    \caption{The features which are included in the performance evaluation
    analysis. Stochastic, makes use of length and makes use of game are APL
    classifiers that determine whether a strategy is stochastic or deterministic,
    whether it makes use of the number of turns or the game's payoffs. The
    memory usage is calculated as the number of turns the strategy considers to
    make an action (which is specified in the APL) divided by the number of
    turns. The SSE (introduced in~\cite{Knight2019}) shows how close a strategy
    is to behaving as a ZDs, and subsequently, in an extortionate way. The
    method identifies the ZDs closest to a given strategy and calculates the
    algebraic distance between them as the sum of squared error (SSE). A SSE value of 1 indicates
    no extortionate behaviour at all whereas a value of 0 indicates that a
    strategy is behaving as a ZDs. The rest of the features considered are the $CC$
    to $C$, $CD$ to $C$, $DC$ to $C$, and $DD$ to $C$ rates as well as
    cooperating ratio of a strategy, the minimum (\(C_{min}\)), maximum
    (\(C_{max}\)), mean (\(C_{mean}\)) and median (\(C_{median}\)) cooperating
    ratios of each tournament.}
    \label{table:manual_features}
\end{table}

The memory usage of strategies is the number of
rounds of play used by the strategy divided by the number of turns in each match.
For example, Winner12 uses the previous two rounds of play, and if participating
in a match with 100 turns its memory usage would be 2/100.
For strategies with an infinite memory size, for example Evolved
FSM 16 Noise 05, memory usage is equal to 1.
Note that for tournaments with a probabilistic
ending the number of turns was not collected, so the memory usage feature is not
used for probabilistic ending tournaments.

The correlation coefficients between the features of
Table~\ref{table:manual_features} the median score and the median normalised
rank are given by Table~\ref{table:correlations}. The correlation coefficients
between all features of Table~\ref{table:manual_features} have been calculated
and a graphical representation can be found in the
Appendix~\ref{app:correlations}.

\newcolumntype{g}{>{\columncolor{Gray}}c}
\begin{table}[!htbp]
    \begin{center}
    \resizebox{.9\textwidth}{!}{
        \begin{tabular}{lggccggccggg}
    \toprule
    &  \multicolumn{2}{g}{Standard} & \multicolumn{2}{c}{Noisy} & \multicolumn{2}{g}{Probabilistic ending} &  \multicolumn{2}{c}{Noisy probabilistic ending} \\
\midrule
{} &  $r$ &  median score &  $r$ &  median score &  $r$ &  median score &  $r$ &  median score \\
\midrule
$CC$ to $C$ rate & -0.501 & 0.501 & -0.210 & 0.194 & -0.336 & 0.348 & 0.087 & 0.015 \\
$CD$ to $C$ rate & 0.226 & -0.199 & 0.337 & -0.235 & 0.458 & -0.352 & 0.609 & -0.372 \\
$DC$ to $C$ rate & 0.127 & -0.100 & 0.227 & -0.111 & 0.164 & -0.105 & 0.410 & -0.203 \\
$DD$ to $C$ rate & 0.412 & -0.396 & 0.549 & -0.391 & 0.433 & -0.378 & 0.615 & -0.407 \\
$C_r$ & -0.323 & 0.383 & 0.298 & -0.051 & -0.060 & 0.160 & 0.595 & -0.213 \\
$C_{max}$ & 0.000& 0.050 & -0.000& 0.244 & -0.000& 0.079 & -0.000& 0.296 \\
$C_{min}$ & 0.000& 0.085 & 0.000& -0.070 & 0.000& 0.128 & 0.000& 0.000\\
$C_{median}$ & 0.000& 0.209 & 0.000& 0.572 & -0.000& 0.324 & 0.000& 0.667 \\
$C_{mean}$ & 0.000& 0.229 & -0.000& 0.583 & -0.000& 0.354 & -0.000& 0.689 \\
$C_r$ / $C_{max}$  & -0.323 & 0.381 & 0.307 & -0.076 & -0.060 & 0.156 & 0.608 & -0.246 \\
$C_{min}$ / $C_r$ & 0.109 & -0.080 & -0.141 & -0.011 & 0.024 & 0.029 & -0.335 & 0.092 \\
$C_r$ / $C_{median}$ & -0.330 & 0.353 & 0.326 & -0.258 & -0.065 & 0.111 & 0.614 & -0.464 \\
$C_r$ / $C_{mean}$ & -0.331 & 0.357 & 0.325 & -0.228 & -0.066 & 0.114 & 0.617 & -0.431 \\
$N$ & -0.000& -0.009 & -0.000& -0.017 & -0.000& 0.011 & 0.000& 0.139 \\
$k$ & -0.000& -0.002 & -0.000& -0.003 & -0.000& 0.010 & -0.000& 0.035 \\
$n$ & -0.000& -0.125 & -0.000& -0.392 & - & - & - & - \\
$p_n$ & - & - & 0.000& -0.244 & - & - & 0.000& -0.272 \\
$p_e$ & - & - & - & - & 0.000& 0.257 & 0.000& 0.568 \\
Make use of game & -0.003 & -0.022 & -0.047 & 0.014 & -0.046 & 0.022 & -0.110 & 0.057 \\
Make use of length & -0.158 & 0.124 & -0.224 & 0.139 & -0.173 & 0.128 & -0.206 & 0.115 \\
SSE & 0.473 & -0.452 & 0.589 & -0.412 & 0.458 & -0.418 & 0.571 & -0.383 \\
stochastic & 0.006 & -0.024 & 0.010 & -0.007 & -0.001 & 0.001 & -0.001 & 0.002 \\
memory usage & -0.098 & 0.108 & -0.080 & 0.114 & - & - & - & - \\
\bottomrule
\end{tabular}
    }
\end{center}
\caption{Correlations between the features of Table~\ref{table:manual_features}
and the normalised rank and the median score.}\label{table:correlations}
\end{table}

In standard tournaments the features $CC$ to $C$, $C_r$, $C_r / C_{\text{max}}$
and the cooperating ratio compared to $C_{\text{median}}$ and $C_{\text{mean}}$
have a moderately negative effect on the normalised rank (smaller rank is better), and a moderate positive
on the median score. The SSE error and the $DD$ to $C$ rate have the opposite
effects. Thus, in standard tournaments behaving cooperatively corresponds to a
more successful performance. Even though being nice generally pays off
that does not hold against defective strategies. Being more cooperative after a mutual
defection, that is not retaliating, is associated to lesser overall success in terms of normalised rank.
Figure~\ref{fig:rates_of_winners_in_standard_tournaments} confirms that the
winners of standard tournaments always cooperate after a mutual cooperation and
almost always defect after a mutual defection.

\begin{figure}[!htbp]
    \centering
    \includegraphics[width=.8\textwidth]{../images/rates_of_winners_in_standard_tournaments.pdf}
    \caption{Distributions of $CC$ to $C$ and $DD$ to $C$ for the winners in
    standard tournaments.}\label{fig:rates_of_winners_in_standard_tournaments}
\end{figure}

Compared to standard tournaments, in both noisy and in probabilistic ending
tournaments the higher the rates of cooperation the lower a strategy's success
and median score. A strategy would want to cooperate less than both
the mean and median cooperator in such settings. In probabilistic ending
tournaments the correlation coefficients have larger values, indicating a
stronger effect. Thus a strategy will be punished more by its cooperative
behaviour in probabilistic ending environments, supporting the results of
Section~\ref{section:evaluation_of_performance}
as well. The distributions of the $C_r$ of the winners in
both tournaments are given by Figure~\ref{fig:c_r_distributions}. It confirms
that the winners in noisy tournaments cooperated less than 35\% of the time
and in probabilistic ending tournaments less than 10\%.
In noisy probabilistic ending tournaments and over all the tournaments' results,
the only features that had a moderate effect are $C_r/C_{\text{mean}},
C_r/C_{\text{max}}$ and $C_r$. In such environments cooperative behaviour
appears to be punished less than in noisy and probabilistic ending
tournaments.


\begin{figure}[!htbp]
    \centering
    \includegraphics[width=.8\textwidth]{../images/c_r_winners_tournaments.pdf}
    \caption{$C_r$ distributions of the winners in noisy and in probabilistic
    ending tournaments.}\label{fig:c_r_distributions}
\end{figure}

A multivariate linear regression has been fitted to model the relationship between
the features and the normalised rank. Based on the graphical representation of
the correlation matrices given in Appendix~\ref{app:correlations} several of the
features are highly correlated and have been removed
before fitting the linear regression model. The features included are given
by Table~\ref{table:linear_regression} alongside their corresponding \(p\) values
in the distinct tournaments and their regression coefficients.

\newcolumntype{g}{>{\columncolor{Gray}}c}
\begin{table}[h]
    \begin{center}
\resizebox{\textwidth}{!}{
    \begin{tabular}{lggccggccgg}
\toprule
& \multicolumn{2}{g}{Standard} & \multicolumn{2}{c}{Noisy} & \multicolumn{2}{g}{Probabilistic ending} & \multicolumn{2}{c}{Noisy probabilistic ending} & \multicolumn{2}{g}{Overall}\\
\midrule
& \multicolumn{2}{g}{\(R\) adjusted: 0.541} & \multicolumn{2}{c}{\(R\) adjusted: 0.639} & \multicolumn{2}{g}{\(R\) adjusted: 0.587} & \multicolumn{2}{c}{\(R\) adjusted: 0.577} & \multicolumn{2}{g}{\(R\) adjusted: 0.242} \\
{} &  Coefficient &  \(p\)-value &  Coefficient &  \(p\)-value &  Coefficient &  \(p\)-value &  Coefficient &  \(p\)-value &  Coefficient &  \(p\)-value \\
\midrule
$CC$ to $C$ rate   & -0.042 &  0.000 & -0.007 &  0.000 &  0.017 &  0.000 &  0.111 &    0.0 & -0.099 &    0.0 \\
$CD$ to $C$ rate   &  0.297 &  0.000 & -0.068 &  0.000 &  0.182 &  0.000 &  0.023 &    0.0 &  0.129 &    0.0 \\
$C_r$ / $C_{max}$  &      - &      - &  1.856 &  0.000 &      - &      - &  1.256 &    0.0 &     - &      - \\
$C_r$ / $C_{mean}$ & -0.468 &  0.000 & -0.577 &  0.000 &  0.525 &  0.000 & -0.120 &    0.0 &  0.300 &    0.0\\
$C_{max}$          & -0.071 &  0.000 &      - &      - & -0.022 &  0.391 &  1.130 &    0.0 &   - &      -   \\
$C_{mean}$         &  0.118 &  0.000 & -2.558 &  0.000 & -0.023 &  0.001 & -1.489 &    0.0 &   - &      -   \\
$C_{min}$          & -0.161 &  0.000 & -1.179 &  0.000 & -0.170 &  0.000 &      - &      - &   - &      -   \\
$C_{min}$ / $C_r$  &  0.057 &  0.000 & -0.320 &  0.000 &  0.125 &  0.000 &      - &      - & -0.103 &    0.0 \\
$DC$ to $C$ rate   &  0.198 &  0.000 &  0.040 &  0.000 & -0.030 &  0.000 &  0.022 &    0.0 &  0.064 &    0.0 \\
$k$                &  0.000 &  0.319 &  0.000 &  0.020 &  0.000 &  0.002 &  0.000 &    0.0 &   - &      -   \\
$n$                &  0.000 &  0.000 &      - &      - &      - &      - &      - &      - &   - &      -   \\
$p_e$              &      - &      - &      - &      - &  0.000 &  0.847 & -0.083 &    0.0 &   - &      -   \\
$p_n$              &      - &      - & -0.048 &  0.000 &      - &      - &      - &      - &   - &      -   \\
SSE                &  0.258 &  0.000 &  0.153 &  0.000 & -0.041 &  0.000 &  0.100 &    0.0 &  0.056 &    0.0 \\
constant           &  0.697 &  0.000 &  1.522 &  0.000 & -0.057 &  0.019 & -0.472 &    0.0 &  0.178 &    0.0 \\
memory usage       & -0.010 &  0.000 & -0.000 &  0.035 &      - &      - &      - &      - &  - &      -   \\
\bottomrule
\end{tabular}}
    \end{center}
    \caption{Results of multivariate linear regressions with \(r\) as the dependent variable.
    \(R\) squared is reported for each model.}
    \label{table:linear_regression}
\end{table}

A multivariate linear regression has also be fitted on the median score. The
coefficients and \(p\) values of the features can be found in
Appendix~\ref{app:regression_median_score}. This approach leads to similar conclusions.

The feature \(C_{r} / C_{\text{mean}}\) has a statistically significant effect
across all models and a high regression coefficient. It has both a positive and
negative impact on the normalised rank depending on the environment. For
standard tournaments, Figure~\ref{fig:discussion_standard} gives the
distributions of several features for the winners of standard tournaments. The
\(C_{r} / C_{\text{mean}}\) distribution of the winner is also given in
Figure~\ref{fig:discussion_standard}. A value of \(C_r / C_{\text{mean}} = 1\)
implies that the cooperating ratio of the winner was the same as the mean
cooperating ratio of the tournament, and in standard tournaments, the median is
1. Therefore, an effective strategy in standard tournaments was the mean
cooperator of its respective tournament.

The distributions of SSE and \(CD\) to \(C\) rate for the winners of standard
tournaments are also given in Figure~\ref{fig:discussion_standard}. The SSE
distributions for the winners indicate that the strategy behaved in a ZD way in
several tournaments, however, not constantly. The winners participated in
matches where they did not try to extortionate their opponents. Furthermore, the
\(CD\) to \(C\) distribution indicates that if a strategy were to defect against
the winners the winners would reciprocate on average with a probability of 0.5.

\begin{figure}[!htbp]
    \centering
        \centering
        \includegraphics[width=\textwidth]{../images/standard_discussion.pdf}
        \caption{Distributions of \(C_r / C_{\text{mean}}\), SSE and \(CD\) to \(C\) ratio
        for the winners of standard tournaments. A
        value of \(C_r / C_{\text{mean}} = 1\) imply that the cooperating ratio of the
        winner was the same as the mean cooperating ratio of the tournament. An SSE distribution
        skewed towards 0 indicates a extortionate behaviour by the strategy.}
        \label{fig:discussion_standard}
\end{figure}

Similarly for the rest of the different tournaments types, and the entire data
set the distributions of \(C_r / C_{\text{mean}}\), SSE and \(CD\) to \(C\) ratio
are given by Figures~\ref{fig:discussion_noisy},~\ref{fig:discussion_probend},
\ref{fig:discussion_probend_noisy} and~\ref{fig:discussion_entire_data}.

Based on the \(C_r / C_{\text{mean}}\) distributions the successful strategies
have adapted differently to the mean cooperator depending on the tournament
type. In noisy tournaments where the median of the distribution is at 0.67, and
thereupon the winners cooperated 67\% of the time the mean cooperator did. In
tournaments with noise and a probabilistic ending the winners cooperated 60\%,
whereas in settings that the type of the tournament can vary between all the
types the winners cooperated 67\% of the time the mean cooperator did. Lastly,
in probabilistic ending tournaments above more defecting
strategies prevail (Section~\ref{section:top_performances}), and this result is
reflected here.

\begin{figure}[!htbp]
    \centering
        \centering
        \includegraphics[width=\textwidth]{../images/noisy_discussion.pdf}
        \caption{Distributions of \(C_r / C_{\text{mean}}\), SSE and \(CD\) to \(C\) ratio
        for the winners of noisy tournaments.}
        \label{fig:discussion_noisy}
\end{figure}

The probability of noise has been observed to substantially affect optimal
behaviour.
Figure~\ref{fig:compared_to_mean_over_noise_probability} gives the ratio \(C_r /
C_{\text{mean}}\) for the winners in tournaments with noise, over the
probability of noise. From Figure~\ref{fig:noisy_discussion_over_noise} it is
clear that the cooperating only 67\% of the time the mean cooperator did is
optimal only when \(p_n \in [0.2, 0.4)\) and \(p_n \in [0.6, 0.7]\). In
environments with \(p_n < 0.1\) the winners want to be close to the mean
cooperator, similarly to standard tournaments, and as the probability of noise
is exceeding 0.5 (where the game is effectively inverted) strategies should
aim to be less and less cooperative.

Figure~\ref{fig:compared_to_mean_over_noise_probability} gives \(C_r /
C_{\text{mean}}\) for the winners over \(p_n\) in tournaments with noise and a
probabilistic ending. The optimal proportions of cooperations are different
now that the number of turns is not fixed, successful strategies
want to be more defecting that the mean cooperator, that only changes when
\(p_n\) approaches 0.5. Figure~\ref{fig:compared_to_mean_over_noise_probability}
demonstrates how the adjustments to \(C_r /C_{\text{mean}}\) change over the
noise in the to the environment, and thus supports how important adapting to
the environment is for a strategy to be successful.

\begin{figure}[!htbp]
    \centering
    \begin{subfigure}{0.485\textwidth}
        \centering
        \includegraphics[width=\textwidth]{../images/noisy_discussion_over_noise.pdf}
        \caption{\(C_r / C_{\text{mean}}\) distribution for winners in noisy tournaments over
        \(p_n\).}\label{fig:noisy_discussion_over_noise}
    \end{subfigure}
    \hfill
    \begin{subfigure}{0.485\textwidth}
        \centering
        \includegraphics[width=\textwidth]{../images/noisy_probend_discussion_over_noise.pdf}
        \caption{\(C_r / C_{\text{mean}}\) distribution for winners in noisy probabilistic ending tournaments over
        \(p_n\).}\label{fig:noisy_probend_discussion_over_noise}
    \end{subfigure}
    \caption{\(C_r / C_{\text{mean}}\) distributions over intervals of \(p_n\).
    These distributions model the optimal proportion of cooperation
    compared to \(C_{\text{mean}}\) as a function of (\(p_n\)).}
    \label{fig:compared_to_mean_over_noise_probability}
\end{figure}

The distributions of the SSE across the tournament types suggest that successful
strategies exhibit some extortionate behaviour, but not constantly.
ZDs are a set of strategies that are often envious as they try to exploit their
opponents. The winners of the tournaments considered in this work are
envious, but not as much as many ZDs.
Though the exact interactions between the matches have not been recorded here,
the work of~\cite{Harper2017} which introduced the trained strategies that
appeared in the top ranked strategies of Section~\ref{section:top_performances}
did. In~\cite{Harper2017} it was shown that clever strategies managed to achieve
mutual cooperation with stronger strategies whilst exploiting the weaker
strategies. This could explain the clever winners
of our analysis, and would explain the SSE distributions. This could also
be the reason why ZDs fail to appear in the tops ranks -- they try to exploit
all opponents and cannot actively adapt back to mutual cooperation against
stronger strategies, which requires more depth of memory. Note that
ZDs also tend to perform poorly in population games for a similar reason: they
attempt to exploit other players using ZDs, failing to form a cooperative
subpopulation~\cite{Knight2017evolution}. This makes them good invaders but poor resisters of invasion.

\begin{figure}[!htbp]
    \centering
        \centering
        \includegraphics[width=\textwidth]{../images/probend_discussion.pdf}
        \caption{Distributions of \(C_r / C_{\text{mean}}\), SSE and \(CD\) to \(C\) ratio
        for the winners of probabilistic ending tournaments.}
        \label{fig:discussion_probend}
\end{figure}

The distributions of the \(CD\) to \(C\) rate evaluate the behaviour of a
successful strategy after its opponent has defected against it. In standard
tournaments it was observed that a successful strategy reciprocates with a
probability of 0.5, and in a setting that the type
of the tournament can vary between all the examined types a winning strategy
would reciprocate on average with a probability of 0.58. In
tournaments with noise a strategy is less likely to cooperate following a
defection compared to standard tournaments, and in probabilistic ending
tournaments a strategy will reciprocate a defection.
This leads to adjusting the recommendation of being provocable to defections made
by R. Axerlod. A strategy should be provocable in tournaments with short matches,
but in the rest of the settings a strategy should be more generous.

\begin{figure}[!htbp]
    \centering
        \centering
        \includegraphics[width=\textwidth]{../images/probend_noisy_discussion.pdf}
        \caption{Distributions of \(C_r / C_{\text{mean}}\), SSE and \(CD\) to \(C\) ratio
        for the winners of noisy probabilistic ending tournaments.}
        \label{fig:discussion_probend_noisy}
\end{figure}

Further statistically significant features with strong effects include \(C_r /
C_{\text{min}}\), \(C_r / C_{\text{max}}\), \(C_{\text{min}}\) and
\(C_{\text{max}}\). These add more emphasis on how important it is for a  a
strategy to adapt to its environment. Finally, the features number of turns,
repetitions and the probabilities of noise and the game ending had no
significant effects based on the multivariate regression models.

\begin{figure}[!htbp]
    \centering
        \centering
        \includegraphics[width=\textwidth]{../images/entire_data_discussion.pdf}
        \caption{Distributions of \(C_r / C_{\text{mean}}\), SSE and \(CD\) to \(C\) ratio
        for the winners over the tournaments of the entire data set.}
        \label{fig:discussion_entire_data}
\end{figure}

A third method that evaluates the importance of the features in
Table~\ref{table:manual_features} using clustering and random forests can be found in
the Appendix~\ref{app:clustering}. The results uphold the outcomes of the
correlation and multivariate regression. It also evaluates the effects
of the whether or not a strategy is stochastic, makes use of the knowledge of the utility values, or makes use of match length. These were not evaluated by the methods above because there are binary variables.
The results showed that they have no significant effect on a strategy's
performance.

\section{Discussion}\label{section:conclusion}

This manuscript explored the performance of \numberofstrategies strategies of
the IPD in \numberofalltournaments computer tournaments. The collection of
computer tournaments presented here is the largest and most diverse collection in the
literature. The \numberofstrategies strategies are drawn from the APL and include
strategies from the IPD literature. The computer tournaments include tournaments of
four different types.

So what is the best way of playing the IPD? And is
there a single dominant strategy for the IPD? 

There was not a single strategy
within the collection of the \numberofstrategies strategies that managed to
perform well in all the tournaments variations it competed in. Even if on
average a strategy ranked highly in a specific environment this did not
guarantee its success over the different tournament types. Nevertheless, in
Sections~\ref{section:top_performances}
and~\ref{section:evaluation_of_performance} we examined the best performing
strategies across various tournament types and analysed their salient features.
this demonstrated that there are properties associated with the success of
strategies which in fact contradict the originally suggested properties of R.
Axelrod~\cite{Axelrod1981}.

We showed that complex or \textbf{clever} strategies can be effective, whether
trained against a corpus of possible opponents or purposely designed to mitigate
the impact of noise such as the DBS strategy. Moreover, we found some
strategies designed or trained for noisy environments were also highly ranked in
noise-free tournaments which reinforces the idea that strategies'
complexity/cleverness is not necessarily a liability, rather it can confer
adaptability to a more diverse set of environments.
We also showed that while the type of exploitation attempted by ZDs is
not typically effective in standard tournaments, \textbf{envious} strategies
capable of both exploiting and not their opponents can be highly successful.
Based on the results of~\cite{Harper2017} this could be because they are
selectively exploiting weaker opponents while mutually cooperating with stronger
opponents. Highly noisy or tournaments with short matches also favoured envious
strategies. These environments mitigated the value of being nice. Uncertainty
enables exploitation, reducing the ability of maintaining or enforcing mutual
cooperation, while triggering grudging strategies to switch from typically
cooperating to typically defecting.

The features analysis of the best performing strategies demonstrated that a
strategy should reciprocate, as suggested by R. Axelrod, but it should relax its
readiness to do so and be more \textbf{generous}. For noisy environments this is
inline with the results of~\cite{Bendor1991, Donninger1986, Molander1985,
Hammerstein1984}, however, we also showed that generosity pays off even in
standard settings, and that in fact the only setting a strategy would want to be
too provocable is when the matches are not long. Forgiveness as defined by R.
Axerlod was not explored here. This was mainly because the two round states were
not recorded during the data collection. This could be a topic of future work
that examines the impact of considering more rounds of history.
The features analysis also concluded that
there is a significant importance in \textbf{adapting to the environment}, and
more specifically, to the mean cooperator. 
In standard tournaments a strategy would aim to be the
mean cooperator while in noisy tournaments the best performing players
cooperate at a lower rate than the tournament population on average.
Moreover, the manner in which a
strategy achieves a given cooperation rate relative to the tournament population
average is important.

This could potentially explain the early success of TFT. TFT naturally achieves
a cooperation rate near $C_{\text{mean}}$ by virtue of copying its opponent's
last move while also minimizing instances where it is exploited by an opponent
(cooperating while the opponent defects), at least in non-noisy tournaments. It
could also explain why Tit For \(N\) Tats does not fare well for $N > 1$ -- it
fails to achieve the proper cooperation ratio by tolerating too many defections.

Similarly, our results could suggest an explanation regarding the intuitively
unexpected effectiveness of memory-one strategies historically. Given that among
the important features associated with success are the relative cooperation rate
to the population average and the four memory-one probabilities of cooperating
conditional on the previous round of play, these features can be optimized by a
memory-one strategy such as TFT. Usage of more history becomes valuable when
there are exploitable opponent patterns. This is indicated by the importance of
SSE as a feature, showing that the first-approximation provided by a memory-one
strategy is no longer sufficient.

These results highlight a central idea in evolutionary game theory in this
context: the fitness landscape is a function of the population (where fitness in
this case is tournament performance). While that may seem obvious now, it shows
why historical tournament results on small or arbitrary populations of
strategies have so often failed to produce generalizable results.

Overall, the five properties successful strategies need to have in a IPD competition
based on the analysis that has been presented in this manuscript are:

\begin{itemize}
    \item Be ``nice'' in non-noisy environments or when game lengths are longer
    \item Be provocable in tournaments with short matches, and generous when matches are longer
    \item Be a little bit envious
    \item Be clever
    \item Adapt to the environment (including the population of strategies).
\end{itemize}

The data set described in this work contains the largest number of IPD tournaments,
to the authors knowledge. The raw data set is available at~\cite{raw_data} and the
processed data at~\cite{data}. Further data mining
could be applied and provide new insights in the field.

\bibliographystyle{plain}
\bibliography{bibliography}

\section{Acknowledgements}

A variety of software have been used in this work:

\begin{itemize}
    \item The Axelrod-Python library for IPD simulations~\cite{axelrodproject}.
    \item The Matplotlib library for visualisation~\cite{hunter2007matplotlib}.
    \item The Numpy library for data manipulation~\cite{walt2011numpy}.
    \item The scikit-learn library for data analysis~\cite{scikit-learn}.
\end{itemize}



\end{document}
