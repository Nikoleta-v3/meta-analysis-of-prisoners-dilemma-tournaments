\section{Top ranked strategies and features analysis for the entire data set}\label{app:analysis_full_dataset}

In this section we carry out a similar analysis as the one presented in
main manuscript, but this time we use the entire data set for noisy and
probabilistic ending tournaments.

\subsection{Top ranked strategies across tournaments}

The top 15 strategies for each tournament type based on \(\bar{r}\) are given in
Table~\ref{table:top_performances}. The data collection process was designed
such that the probabilities of noise and ending of the match varied between 0
and 1. However, commonly used values for these probabilities are values less
than 0.1. Thus, Table~\ref{table:top_performances} also includes the top 15
strategies in noisy tournaments with \(p_n < 0.1\) and probabilistic ending
tournaments with \(p_e < 0.1\).

\newcolumntype{g}{>{\columncolor{Gray}}l}
\begin{table}[!htbp]
    \begin{center}
    \resizebox{\textwidth}{!}{
        \input{top_perfomances_full_data_set.tex}
    }
\end{center}
\caption{Top performances for each tournament type based on $\bar{r}$. The
results of each type are based on 11420 unique tournaments. The
results for noisy tournaments with \(p_n < 0.1\) are based on 1151 tournaments,
and for probabilistic ending tournaments with \(p_e < 0.1\) on 1139. The top
ranks indicate that trained strategies perform well in a variety of
environments, but so do simple deterministic strategies. The normalised medians
are close to 0 for most environments, except environments with noise not
restricted to 0.1 regardless of the number of turns. Noisy and noisy probabilistic
ending tournaments have the highest medians.}
\label{table:top_performances}
\end{table}

The \(r\) distributions for the top ranked strategies of Table~\ref{table:top_performances}
are given by Figure~\ref{fig:r_distributions}.

\begin{figure*}[!htbp]
    \centering
    \begin{subfigure}{0.485\textwidth}
        \centering
        \includegraphics[width=\textwidth]{../images/r_distribution_standard.pdf}
        \caption{$r$ distributions of top 15 strategies in standard tournaments.}\label{fig:std_results}
    \end{subfigure}
    \hfill
    \begin{subfigure}{0.485\textwidth}
        \centering
        \includegraphics[width=\textwidth]{../images/r_distribution_noise_subset.pdf}
        \caption{$r$ distributions of top 15 strategies in noisy tournaments with \(p_n < 0.1\).}\label{fig:noise_subset_results}
    \end{subfigure}
    \vskip\baselineskip
    \begin{subfigure}{0.485\textwidth}
        \centering
        \includegraphics[width=\textwidth]{../images/r_distribution_noise.pdf}
        \caption{$r$ distributions of top 15 strategies in noisy tournaments.}\label{fig:noise_results}
    \end{subfigure}
    \quad
    \begin{subfigure}{0.485\textwidth}
        \centering
        \includegraphics[width=\textwidth]{../images/r_distribution_probend_subset.pdf}
        \caption{\(r\) distributions of top 15 strategies in 1139 probabilistic ending
        tournaments with \(p_e < 0.1\).}
        \label{fig:probend_subset_results}
    \end{subfigure}
    \vskip\baselineskip
    \begin{subfigure}{0.485\textwidth}
        \centering
        \includegraphics[width=\textwidth]{../images/r_distribution_probend.pdf}
        \caption{$r$ distributions of top 15 strategies in probabilistic ending tournaments.}\label{fig:probend_results}
    \end{subfigure}
    \quad
    \begin{subfigure}{0.485\textwidth}
        \centering
        \includegraphics[width=\textwidth]{../images/r_distribution_probend_noise.pdf}
        \caption{$r$ distributions of top 15 strategies in noisy probabilistic ending tournaments.}
        \label{fig:probend_noise_results}
    \end{subfigure}
    \caption{\(r\) distributions of the top 15 strategies in different
    environments. A lower value of \(\bar{r}\) corresponds to a more successful
    performance. A strategy's \(r\) distribution skewed towards zero indicates
    that the strategy ranked highly in most tournaments it participated in. Most
    distributions are skewed towards zero except the distributions with
    unrestricted noise, supporting the conclusions from
    Table~\ref{table:top_performances}.}\label{fig:r_distributions}
\end{figure*}

In standard tournaments 10 out of the 15 top strategies were introduced
in~\cite{Harper2017}. These are strategies based on finite state automata (FSM),
hidden Markov models (HMM), artificial neural networks (ANN), lookup tables
(LookerUp) and stochastic lookup tables (Gambler) that have been trained using
reinforcement learning algorithms (evolutionary and particle swarm algorithms).
They have been trained to perform well against a subset of the strategies
in APL in a standard tournament, thus their performance in the
specific setting was anticipated although still noteworthy given the random
sampling of tournament participants. DoubleCrosser, BackStabber and Fool Me Once, are
strategies not from the literature but from the APL. DoubleCrosser is an extension
of BackStabber and both strategies make use of the number of turns because they are
set to defect on the last two rounds. It should be noted that these
strategies can be characterised as ``cheaters'' because the source code of the strategies
allows them to know the number of turns in a match (unless the match has a probabilistic ending). These strategies were expected to not perform as well in
tournaments where the number of turns is not specified. Finally, Winner
12~\cite{mathieu2017} and DBS~\cite{Au2006} are both from the literature.
DBS is a strategy specifically designed for noisy environments, however, it ranks
highly in standard tournaments as well. Similarly the fourth ranked player,
Evolved FSM 16 Noise 05, was
trained for noisy tournaments yet performs well in standard tournaments.
Figure~\ref{fig:std_results} shows that these strategies typically perform
well in any standard tournament in which they participate.

In the case of noisy tournaments with smaller noise \(p_n < 0.1\) the top
performing strategies
include strategies specifically designed for noisy tournaments. These are DBS,
Evolved FSM 16 Noise 05, Evolved ANN 5 Noise 05, PSO Gambler 2 2 2 Noise 05 and
Omega Tit For Tat~\cite{kendall2007iterated}. Omega TFT, a strategy designed
to break the deadlocking cycles of \(CD\) and \(DC\) that TFT can fall into in noisy
environments, places 10th. The rest of the top ranks are
occupied by strategies which performed well in standard tournaments and
deterministic strategies such as Spiteful Tit For Tat~\cite{prison}, Level
Punisher~\cite{Eckhart2015}, Eugine Nier~\cite{lesswrong}.

In contrast, the performance of the top ranked strategies in noisy environments
when \(p_n\in [0, 1]\) is bimodal. The top strategies include strategies which
decide their actions based on the cooperation to defection ratio, such as
ShortMem~\cite{Andre2013}, Grumpy~\cite{axelrodproject} and
e~\cite{axelrodproject}, and the Retaliate strategies which are designed to
defect if the opponent has tricked them more often than a given percentage of the times that
they have done the same. The bimodality of the \(r\) distributions is explained
by Figure~\ref{fig:effect_of_noise} which demonstrates that the top 6 strategies
were highly ranked due to the their performance in tournaments with \(p_n>0.5\),
and that in tournaments with \(p_n<0.5\) they
performed poorly. At a noisy level of \(0.5\) or greater, mostly cooperative strategies
become mostly defectors and vice versa.

\begin{figure}[!htbp]
    \centering
    \includegraphics[width=.92\textwidth]{../images/noise_effect.pdf}
    \caption{Normalised rank \(r\) distributions for top 6 strategies in noisy tournaments over
    the probability of noisy ($p_n$).}
    \label{fig:effect_of_noise}
\end{figure}

The most effective strategies in probabilistic ending
tournaments with \(p_e< 0.1\) are a series of ensemble Meta strategies, trained strategies
which performed well
in standard tournaments, and Grudger~\cite{axelrodproject} and Spiteful Tit for
Tat~\cite{prison}. The Meta strategies~\cite{axelrodproject} utilize a team of
strategies and aggregate the potential actions of the team members into a single action
in various ways. Figure~\ref{fig:probend_subset_results} indicates that these strategies
performed well in any probabilistic ending tournament.

In probabilistic ending tournaments with \(p_e \in [0, 1]\) the top ranks are
mostly occupied by defecting strategies such as Better and Better, Gradual
Killer, Hard Prober (all from~\cite{axelrodproject}), Bully (Reverse Tit For
Tat)~\cite{Nachbar1992} and Defector, and a series of strategies based on finite
state automata introduced by Daniel Ashlock and Wendy Ashlock: Fortress 3,
Fortress 4 (both introduced in~\cite{Ashlock2006}), Raider~\cite{Ashlock2014}
and Solution B1~\cite{Ashlock2014}. The success of defecting strategies in
probabilistic ending tournaments is due to larger values of
\(p_e\) which lead to shorter matches (the expected number of rounds is \(1 / p_e\)), so the
impact of the PD being iterated is subdued. This is captured by the Folk
Theorem~\cite{Fudenberg2009} as defecting strategies do better when the likelihood
of the game ending in the next turn increases.
This is demonstrated by Figure~\ref{fig:effect_of_probend}, which gives the
distributions of \(r\) for the top 6 strategies in probabilistic ending tournaments
over \(p_e\).

\begin{figure}[!htbp]
    \centering
    \includegraphics[width=.92\textwidth]{../images/folk_theorem.pdf}
    \caption{Normalised rank \(r\) distributions for top 6 strategies in probabilistic ending tournaments
    over $p_e$. The 6 strategies start of with a high median rank,
    however, their ranked decreased as the the probability of the game ending
    increased and at the point of \(p_e = 0.1\).}
    \label{fig:effect_of_probend}
\end{figure}

The top performances in tournaments with both noise and a probabilistic ending
have the largest median values compared to the top rank strategies of the other
tournament types. The \(\bar{r}\) for the top strategy is approximately at 0.3,
indicating that the most successful strategy can on average just place in the
top 30\% of the competition.

On the whole, the analysis has shown that:

\begin{itemize}
    \item In standard tournaments the dominating strategies were
    strategies that had been trained using reinforcement learning techniques.
    \item In noisy environments where the noise probability strictly less than
    0.1 was considered, the successful strategies were strategies specifically
    designed or trained for noisy environments.
    \item In probabilistic ending tournaments most of the highly ranked
    strategies were defecting strategies and trained finite state automata, all
    by the authors of~\cite{Ashlock2006, Ashlock2014}. These strategies ranked
    high due to their performance in tournaments where the probability of the
    game ending after each turn was bigger than 0.1.
    \item In probabilistic tournaments with \(p_e\) less than 0.1 the highly
    ranked strategies were strategies based on the behaviour of others.
    \item From the collection of strategies considered here,  no strategy can be
    consistently successful in noisy environments, except if the value of noise
    is constrained to less than a 0.1.
\end{itemize}

\section{The effect of strategy features on performance}

The correlation coefficients between the strategies' features and the median
score and the median normalised rank for the full dataset of tournaments are
given by Table~\ref{table:correlations}. The correlation coefficients between
all features have been calculated and a graphical representation can be found in
the Section~\ref{app:correlations}.

\newcolumntype{g}{>{\columncolor{Gray}}c}
\begin{table}[!htbp]
    \begin{center}
    \resizebox{.9\textwidth}{!}{
        \begin{tabular}{lggccggccggg}
    \toprule
    &  \multicolumn{2}{g}{Standard} & \multicolumn{2}{c}{Noisy} & \multicolumn{2}{g}{Probabilistic ending} &  \multicolumn{2}{c}{Noisy probabilistic ending} \\
\midrule
{} &  $r$ &  median score &  $r$ &  median score &  $r$ &  median score &  $r$ &  median score \\
\midrule
$CC$ to $C$ rate & -0.501 & 0.501 & 0.413 & -0.504 & 0.408 & -0.323 & 0.260 & 0.023 \\
$CD$ to $C$ rate & 0.226 & -0.199 & 0.457 & -0.331 & 0.320 & -0.017 & 0.205 & -0.220 \\
$DC$ to $C$ rate & 0.127 & -0.100 & 0.509 & -0.504 & -0.018 & 0.033 & 0.341 & -0.016 \\
$DD$ to $C$ rate & 0.412 & -0.396 & 0.533 & -0.436 & -0.103 & 0.176 & 0.378 & -0.263 \\
$C_r$ & -0.323 & 0.383 & 0.711 & -0.678 & 0.714 & -0.832 & 0.579 & -0.136 \\
$C_{max}$ & 0.000& 0.050 & 0.000& 0.023 & -0.000& 0.046 & 0.000& -0.004 \\
$C_{min}$ & 0.000& 0.085 & -0.000& -0.017 & -0.000& 0.007 & -0.000& 0.041 \\
$C_{median}$ & 0.000& 0.209 & -0.000& 0.240 & 0.000& 0.187 & 0.000& 0.673 \\
$C_{mean}$ & 0.000& 0.229 & -0.000& 0.271 & 0.000& 0.200 & -0.000& 0.690 \\
$C_r$ / $C_{max}$  & -0.323 & 0.381 & 0.616 & -0.551 & 0.715 & -0.833 & 0.536 & -0.117 \\
$C_{min}$ / $C_r$ & 0.109 & -0.080 & -0.358 & 0.250 & -0.134 & 0.151 & -0.368 & 0.113 \\
$C_r$ / $C_{median}$ & -0.330 & 0.353 & 0.652 & -0.669 & 0.713 & -0.852 & 0.330 & -0.466 \\
$C_r$ / $C_{mean}$ & -0.331 & 0.357 & 0.731 & -0.740 & 0.721 & -0.861 & 0.650 & -0.621 \\
$N$ & -0.000& -0.009 & 0.000& 0.002 & 0.000& 0.003 & 0.000& 0.001 \\
$k$ & -0.000& -0.002 & 0.000& 0.002 & 0.000& 0.001 & 0.000& -0.008 \\
$n$ & -0.000& -0.125 & -0.000& -0.024 & - & - & - & - \\
$p_n$ & - & - & 0.000& 0.207 & - & - & 0.000& -0.650 \\
$p_e$ & - & - & - & - & 0.000& 0.165 & 0.000& -0.058 \\
Make use of game & -0.003 & -0.022 & 0.025 & -0.082 & -0.053 & -0.108 & 0.013 & -0.016 \\
Make use of length & -0.158 & 0.124 & 0.005 & -0.123 & -0.025 & -0.090 & 0.014 & -0.016 \\
SSE & 0.473 & -0.452 & 0.463 & -0.337 & -0.157 & 0.224 & 0.305 & -0.259 \\
stochastic & 0.006 & -0.024 & 0.022 & -0.026 & 0.002 & -0.130 & 0.021 & -0.013 \\
memory usage & -0.098 & 0.108 & -0.009 & -0.017 & - & - & - & - \\
\bottomrule
\end{tabular}
    }
\end{center}
\caption{Correlations between the strategies' features
and the normalised rank and the median score.}\label{table:correlations}
\end{table}

In standard tournaments the features $CC$ to $C$, $C_r$, $C_r / C_{\text{max}}$
and the cooperating ratio compared to $C_{\text{median}}$ and $C_{\text{mean}}$
have a moderately negative effect on the normalised rank (smaller rank is
better), and a moderate positive on the median score. The SSE error and the $DD$
to $C$ rate have the opposite effects. Thus, in standard tournaments behaving
cooperatively corresponds to a more successful performance. Even though being
nice generally pays off that does not hold against defective strategies. Being
more cooperative after a mutual defection, that is not retaliating, is
associated to lesser overall success in terms of normalised rank.
Figure~\ref{fig:rates_of_winners_in_standard_tournaments} confirms that the
winners of standard tournaments always cooperate after a mutual cooperation and
almost always defect after a mutual defection.

\begin{figure}[!htbp]
    \centering
    \includegraphics[width=.8\textwidth]{../images/rates_of_winners_in_standard_tournaments.pdf}
    \caption{Distributions of $CC$ to $C$ and $DD$ to $C$ for the winners in
    standard tournaments.}\label{fig:rates_of_winners_in_standard_tournaments}
\end{figure}

Compared to standard tournaments, in both noisy and in probabilistic ending
tournaments the higher the rates of cooperation the lower a strategy's success
and median score. A strategy would want to cooperate less than both the mean and
median cooperator in such settings. In probabilistic ending tournaments the
correlation coefficients have larger values, indicating a stronger effect. Thus
a strategy will be punished more by its cooperative behaviour in probabilistic
ending environments, supporting the results of the previous subsection as well.
The distributions of the $C_r$ of the winners in both tournaments are given by
Figure~\ref{fig:c_r_distributions}. It confirms that the winners in noisy
tournaments cooperated less than 35\% of the time and in probabilistic ending
tournaments less than 10\%. In noisy probabilistic ending tournaments and over
all the tournaments' results, the only features that had a moderate effect are
$C_r/C_{\text{mean}}, C_r/C_{\text{max}}$ and $C_r$. In such environments
cooperative behaviour appears to be punished less than in noisy and
probabilistic ending tournaments.


\begin{figure}[!htbp]
    \centering
    \includegraphics[width=.8\textwidth]{../images/c_r_winners_tournaments.pdf}
    \caption{$C_r$ distributions of the winners in noisy and in probabilistic
    ending tournaments.}\label{fig:c_r_distributions}
\end{figure}

A multivariate linear regression has been fitted to model the relationship between
the features and the normalised rank. Based on the graphical representation of
the correlation matrices given in Section~\ref{app:correlations} several of the
features are highly correlated and have been removed
before fitting the linear regression model. The features included are given
by Table~\ref{table:linear_regression} alongside their corresponding \(p\) values
in the distinct tournaments and their regression coefficients.

\newcolumntype{g}{>{\columncolor{Gray}}c}
\begin{table}[h]
    \begin{center}
\resizebox{\textwidth}{!}{
    \begin{tabular}{lggccggccgg}
\toprule
& \multicolumn{2}{g}{Standard} & \multicolumn{2}{c}{Noisy} & \multicolumn{2}{g}{Probabilistic ending} & \multicolumn{2}{c}{Noisy probabilistic ending} \\
\midrule
& \multicolumn{2}{g}{\(R\) adjusted: 0.541} & \multicolumn{2}{c}{\(R\) adjusted: 0.588} & \multicolumn{2}{g}{\(R\) adjusted: 0.587} & \multicolumn{2}{c}{\(R\) adjusted: 0.471} \\
{} &  Coefficient &  \(p\)-value &  Coefficient &  \(p\)-value &  Coefficient &  \(p\)-value &  Coefficient &  \(p\)-value \\
\midrule
constant           &  0.695 &  0.000 &  0.443 &    0.0 & -0.057 &  0.018 &  0.004 &  0.031 \\
$CC$ to $C$ rate   & -0.042 &  0.000 &  0.150 &    0.0 &  0.017 &  0.000 &  0.197 &  0.000 \\
$CD$ to $C$ rate   &  0.297 &  0.000 & -0.034 &    0.0 &  0.182 &  0.000 &  0.022 &  0.000 \\
$DC$ to $C$ rate   &  0.198 &  0.000 &  0.064 &    0.0 & -0.030 &  0.000 &  0.090 &  0.000 \\
SSE                &  0.258 &  0.000 &  0.237 &    0.0 & -0.041 &  0.000 &  0.144 &  0.000 \\
$C_{max}$          & -0.068 &  0.000 &      - &      - & -0.021 &  0.403 & -0.090 &  0.000 \\
$C_{min}$          & -0.161 &  0.000 &  1.068 &    0.0 & -0.170 &  0.000 &      - &      - \\
$C_{mean}$         &  0.117 &  0.000 & -0.722 &    0.0 & -0.024 &  0.000 & -0.112 &  0.000 \\
$C_{min}$ / $C_r$  &  0.057 &  0.000 & -0.544 &    0.0 &  0.125 &  0.000 &      - &      - \\
$C_r$ / $C_{mean}$ & -0.468 &  0.000 &  0.272 &    0.0 &  0.525 &  0.000 &  0.403 &  0.000 \\
$k$                &  0.000 &  0.325 &  0.000 &    0.1 &  0.000 &  0.002 &  0.001 &  0.000 \\
$n$                &  0.000 &  0.000 &      - &      - &      - &      - &      - &      - \\
memory usage       & -0.010 &  0.000 &  0.002 &    0.0 &      - &      - &      - &      - \\
$p_n$              &      - &      - & -0.039 &    0.0 &      - &      - &      - &      - \\
$p_e$              &      - &      - &      - &      - &  0.000 &  0.757 & -0.149 &  0.000 \\
\bottomrule
\end{tabular}}
    \end{center}
    \caption{Results of multivariate linear regressions with \(r\) as the dependent variable.
    \(R\) squared is reported for each model.}
    \label{table:linear_regression}
\end{table}

A multivariate linear regression has also be fitted on the median score. The
coefficients and \(p\) values of the features can be found in
Section~\ref{app:regression_median_score}. This approach leads to similar conclusions.

The feature \(C_{r} / C_{\text{mean}}\) has a statistically significant effect
across all models and a high regression coefficient. It has both a positive and
negative impact on the normalised rank depending on the environment. For
standard tournaments, Figure~\ref{fig:discussion_standard} gives the
distributions of several features for the winners of standard tournaments. The
\(C_{r} / C_{\text{mean}}\) distribution of the winner is also given in
Figure~\ref{fig:discussion_standard}. A value of \(C_r / C_{\text{mean}} = 1\)
implies that the cooperating ratio of the winner was the same as the mean
cooperating ratio of the tournament, and in standard tournaments, the median is
1. Therefore, an effective strategy in standard tournaments was the mean
cooperator of its respective tournament.

The distributions of SSE and \(CD\) to \(C\) rate for the winners of standard
tournaments are also given in Figure~\ref{fig:discussion_standard}. The SSE
distributions for the winners indicate that the strategy behaved in a ZD way in
several tournaments, however, not constantly. The winners participated in
matches where they did not try to extortionate their opponents. Furthermore, the
\(CD\) to \(C\) distribution indicates that if a strategy were to defect against
the winners the winners would reciprocate on average with a probability of 0.5.

\begin{figure}[!htbp]
    \centering
        \centering
        \includegraphics[width=\textwidth]{../images/standard_discussion.pdf}
        \caption{Distributions of \(C_r / C_{\text{mean}}\), SSE and \(CD\) to \(C\) ratio
        for the winners of standard tournaments. A
        value of \(C_r / C_{\text{mean}} = 1\) imply that the cooperating ratio of the
        winner was the same as the mean cooperating ratio of the tournament. An SSE distribution
        skewed towards 0 indicates a extortionate behaviour by the strategy.}
        \label{fig:discussion_standard}
\end{figure}

Similarly for the rest of the different tournaments types, and the entire data
set the distributions of \(C_r / C_{\text{mean}}\), SSE and \(CD\) to \(C\) ratio
are given by Figures~\ref{fig:discussion_noisy},~\ref{fig:discussion_probend},
\ref{fig:discussion_probend_noisy} and~\ref{fig:discussion_entire_data}.

Based on the \(C_r / C_{\text{mean}}\) distributions the successful strategies
have adapted differently to the mean cooperator depending on the tournament
type. In noisy tournaments where the median of the distribution is at 0.67, and
thereupon the winners cooperated 67\% of the time the mean cooperator did. In
tournaments with noise and a probabilistic ending the winners cooperated 60\%,
whereas in settings that the type of the tournament can vary between all the
types the winners cooperated 67\% of the time the mean cooperator did. Lastly,
in probabilistic ending tournaments above more defecting
strategies prevail (Section~\ref{section:top_performances}), and this result is
reflected here.

\begin{figure}[!htbp]
    \centering
        \centering
        \includegraphics[width=\textwidth]{../images/noisy_discussion.pdf}
        \caption{Distributions of \(C_r / C_{\text{mean}}\), SSE and \(CD\) to \(C\) ratio
        for the winners of noisy tournaments.}
        \label{fig:discussion_noisy}
\end{figure}

The probability of noise has been observed to substantially affect optimal
behaviour.
Figure~\ref{fig:compared_to_mean_over_noise_probability} gives the ratio \(C_r /
C_{\text{mean}}\) for the winners in tournaments with noise, over the
probability of noise. From Figure~\ref{fig:noisy_discussion_over_noise} it is
clear that the cooperating only 67\% of the time the mean cooperator did is
optimal only when \(p_n \in [0.2, 0.4)\) and \(p_n \in [0.6, 0.7]\). In
environments with \(p_n < 0.1\) the winners want to be close to the mean
cooperator, similarly to standard tournaments, and as the probability of noise
is exceeding 0.5 (where the game is effectively inverted) strategies should
aim to be less and less cooperative.

Figure~\ref{fig:compared_to_mean_over_noise_probability} gives \(C_r /
C_{\text{mean}}\) for the winners over \(p_n\) in tournaments with noise and a
probabilistic ending. The optimal proportions of cooperations are different
now that the number of turns is not fixed, successful strategies
want to be more defecting that the mean cooperator, that only changes when
\(p_n\) approaches 0.5. Figure~\ref{fig:compared_to_mean_over_noise_probability}
demonstrates how the adjustments to \(C_r /C_{\text{mean}}\) change over the
noise in the to the environment, and thus supports how important adapting to
the environment is for a strategy to be successful.

\begin{figure}[!htbp]
    \centering
    \begin{subfigure}{0.485\textwidth}
        \centering
        \includegraphics[width=\textwidth]{../images/noisy_discussion_over_noise.pdf}
        \caption{\(C_r / C_{\text{mean}}\) distribution for winners in noisy tournaments over
        \(p_n\).}\label{fig:noisy_discussion_over_noise}
    \end{subfigure}
    \hfill
    \begin{subfigure}{0.485\textwidth}
        \centering
        \includegraphics[width=\textwidth]{../images/noisy_probend_discussion_over_noise.pdf}
        \caption{\(C_r / C_{\text{mean}}\) distribution for winners in noisy probabilistic ending tournaments over
        \(p_n\).}\label{fig:noisy_probend_discussion_over_noise}
    \end{subfigure}
    \caption{\(C_r / C_{\text{mean}}\) distributions over intervals of \(p_n\).
    These distributions model the optimal proportion of cooperation
    compared to \(C_{\text{mean}}\) as a function of (\(p_n\)).}
    \label{fig:compared_to_mean_over_noise_probability}
\end{figure}

The distributions of the SSE across the tournament types suggest that successful
strategies exhibit some extortionate behaviour, but not constantly.
ZDs are a set of strategies that are often envious as they try to exploit their
opponents. The winners of the tournaments considered in this work are
envious, but not as much as many ZDs.

\begin{figure}[!htbp]
    \centering
        \centering
        \includegraphics[width=\textwidth]{../images/probend_discussion.pdf}
        \caption{Distributions of \(C_r / C_{\text{mean}}\), SSE and \(CD\) to \(C\) ratio
        for the winners of probabilistic ending tournaments.}
        \label{fig:discussion_probend}
\end{figure}

The distributions of the \(CD\) to \(C\) rate evaluate the behaviour of a
successful strategy after its opponent has defected against it. In standard
tournaments it was observed that a successful strategy reciprocates with a
probability of 0.5, and in a setting that the type
of the tournament can vary between all the examined types a winning strategy
would reciprocate on average with a probability of 0.58. In
tournaments with noise a strategy is less likely to cooperate following a
defection compared to standard tournaments, and in probabilistic ending
tournaments a strategy will reciprocate a defection.
This leads to adjusting the recommendation of being provocable to defections made
by Axerlod. A strategy should be provocable in tournaments with short matches,
but in the rest of the settings a strategy should be more generous.

\begin{figure}[!htbp]
    \centering
        \centering
        \includegraphics[width=\textwidth]{../images/probend_noisy_discussion.pdf}
        \caption{Distributions of \(C_r / C_{\text{mean}}\), SSE and \(CD\) to \(C\) ratio
        for the winners of noisy probabilistic ending tournaments.}
        \label{fig:discussion_probend_noisy}
\end{figure}

Further statistically significant features with strong effects include \(C_r /
C_{\text{min}}\), \(C_r / C_{\text{max}}\), \(C_{\text{min}}\) and
\(C_{\text{max}}\). These add more emphasis on how important it is for a  a
strategy to adapt to its environment. Finally, the features number of turns,
repetitions and the probabilities of noise and the game ending had no
significant effects based on the multivariate regression models.

\begin{figure}[!htbp]
    \centering
        \centering
        \includegraphics[width=\textwidth]{../images/entire_data_discussion.pdf}
        \caption{Distributions of \(C_r / C_{\text{mean}}\), SSE and \(CD\) to \(C\) ratio
        for the winners over the tournaments of the entire data set.}
        \label{fig:discussion_entire_data}
\end{figure}


A third method that evaluates the importance of features using clustering and
random forests can be found in the Section~\ref{app:clustering}. The results uphold
the outcomes of the correlation and multivariate regression. It also evaluates
the effects of the whether or not a strategy is stochastic, makes use of the
knowledge of the utility values, or makes use of match length. These were not
evaluated by the methods above because there are binary variables. The results
showed that they have no significant effect on a strategy's performance.