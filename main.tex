\documentclass{article}
\usepackage[margin=2.5cm, includefoot, footskip=30pt]{geometry}

\setlength{\parindent}{0em}
\setlength{\parskip}{1em}
\renewcommand{\baselinestretch}{1}

%%%%Packages%%%%
\usepackage{amsmath}
\usepackage{booktabs}
\usepackage{graphics}
\usepackage{multicol}
\usepackage{algorithm,algorithmic}
%%%%%%%%%%%%%%%%%

\setlength{\tabcolsep}{3pt}

\title{Applying modern data analysis techniques to tournament results of the
Iterated Prisoner's Dilemma.}
\date{}

\begin{document}

\maketitle

\section{The Prisoner's Dilemma}

The Prisoner's Dilemma is a two player repeated game commonly known in Game Theory
as the model of cooperative behaviour. Each of the players has two strategies,
to either cooperate (C) or defect (D). The decisions are made simultaneously
and independently. The normal form representation of the game is given by
matrix~\ref{matrix:pd}.

\begin{equation}\label{matrix:pd}
    \begin{bmatrix}
    (R,R) & (S,T)  \\
    (T,S) & (P,P)
    \end{bmatrix}
\end{equation}

where the payoffs \((R, P, S, T)\) are constrained by equations (\ref{eq:constrain_one})
and (\ref{eq:constrain_two}).

\begin{equation}\label{eq:constrain_one}
    T > R > P > S,
\end{equation}

\begin{equation}\label{eq:constrain_two}
    2R > T + S.
\end{equation}

In the one shot game the Nash equilibrium is given at (D, D) and both players
received a payoff of \(P\). Thus, at the point where both players defect.
In the repeated form of the game the behaviour becomes more complicated. The optimal
behaviour of a player in the repeated game, in the Iterated Prisoner's Dilemma,
has been the focus of research for years.

An interesting approach was introduced in 1980 by, where a computer tournament
was used to tackle the same question.

\begin{itemize}
    \item The definition of a strategy
    \item The definition of a tournament.
\end{itemize}

\section{Collecting data}

For performing a large number of computer tournaments the open source package
Axelrod Library is used in this work. The package was introduced in 2015,
it is written in the programming language Python and it allow us to perform
a number of tournaments with different strategies. The payoff values used in
Axelrod are \((3, 1, 0, 5)\) and the following type of tournaments have been performed.

There are several tournament types introduced in the literature that have not been
discussed.

\begin{enumerate}
    \item \textbf{Standard tournament}. A round robin tournament where
    the number of turns for each match can vary between 200 and 1. The tournament
    is repeated between 10 and 100 times.
    \item \textbf{Noisy tournament}. Similar to a standard tournament. A noisy
    tournament in a round robin tournament where noise is introduced. Noise is the
    probability that a players action is flipped. The probability of noise is
    ranging between 0 and 1.
    \item \textbf{Probabilistic ending tournament}. Similar to a standard tournament
    however in a probabilistic ending tournament the number of turns is not specified.
    There is a probability (ranges between 0 and 1) that the match will end in
    the next round. Probabilistic ending tournament will be referred to as
    probend hereupon.
    \item \textbf{Noisy and Probabilistic ending tournament}. A combination of
    noisy and probabilistic ending tournaments.
\end{enumerate}

The process for generating the data set is described by Algorithm~\ref{alg:tournaments}.
Every 20 iterations of a random seed a new sample is chosen. For each sample, for 20
repetitions random numbers of turns, repetitions of the tournament, the probability of
noise in the tournament and the probability of the game ending in after each interaction
are sampled.

For that set of parameters, four types of tournaments (as discussed above) are conducted.
A standard one, a noisy one, a probend one and lastly a probend with noise.

\begin{algorithm}
    \caption{Generating data}
    \label{alg:tournaments}
      \begin{algorithmic}[1]
        \FOR{seed \textbf{in} 10000}
            \IF{$seed \ mod \ 20 =0$}
                \STATE size $\gets$ random  size
                \STATE players $\gets$ random players
             \ELSE
                 \STATE turns $\gets$ random turns
                 \STATE repetitions  $\gets$  random  repetitions
                 \STATE noise  $\gets$ random  noise  probability
                 \STATE end  $\gets$  random end probability
                 \STATE standard results $\gets$ tournament(turns, players, repetitions) 
                 \STATE noise results $\gets$ tournament(turns, players, repetitions, noise)
                 \STATE probend results $\gets$ tournament(players, repetitions, end)
                 \STATE probend noise results $\gets$ tournament(players, repetitions, noise, end)
             \ENDIF
             \RETURN standard, noise, probend, probend noise results
        \ENDFOR
      \end{algorithmic}
\end{algorithm}

Once a tournaments is performed we export a summary of the performance
of each strategy in the tournament. In the following sections we discuss how we
manipulate the result set exported by Axelrod and hold a data analysis.

\section{Data preparation}

The data set used for the study is the combination of all of these summary data sets.
The structure of the data set is shown in Table~\ref{table:result_set}. Each row
represents a player in a given tournament.

Name is the name of the strategy. Stochastic, Memory depth and Use of arguments are
characteristics of the given strategy. The performance of the strategy is reflected
by columns Rate, Median score, Rank and Wins.

The environment and the type of tournaments are captured by columns Noise, Probend,
Repetitions, Size and Turns. Note that when turns are non given the tournament is a
probabilistic ending one, and vise versa. Similarly, when Noise argument in non given
the tournament is known to be standard. Finally, when both Noise and Probend are
non zero then the tournament is an probabilistic and noisy tournament

\begin{table}[htpb]
\centering
\resizebox{\columnwidth}{!}{%
\begin{tabular}{ccccccccccccccccccccccccccccccccc}
\toprule
% columns
Name & Stochastic & \multicolumn{1}{p{2cm}}{\centering Memory \\ depth} &
\multicolumn{1}{p{2cm}}{\centering Use of \\ game} &
\multicolumn{1}{p{2cm}}{\centering Use of \\ length} & CC rate &  CD rate &
\multicolumn{1}{p{2cm}}{\centering Cooperation \\rating} & DC rate & DD rate &
Median score &  Rank &  Wins &  Noise  &  Probend & Repetitions &  Seed & Size & Turns \\
\midrule
% first row
Adaptive & False & inf & True & False & 0.195 &  0.214 & 0.409 & 0.279 & 0.310 &
2.430 & 14 &  77.0 &  0.391 & NaN & 25.0 & 0.0 & 101.0 & 148.0 \\
% second row
Adaptive & False & inf & True & False & 0.783 & 0.194 & 0.928 & 0.012 & 0.009 &
2.292 & 88 & 5.0 &  0.391 & 0.370 & 25.0 & 0.0 & 101.0 & NaN \\
\bottomrule
\end{tabular}}
\caption{Result Set}
\label{table:result_set}
\end{table}

\begin{itemize}
    \item the cooperation in each environment
    \item show that got a uniform distribution of our parameters
\end{itemize}
In this work each tournament type is studied individually. An individual data set
containing the following information is constructed for each type.

\begin{multicols}{2}
    \begin{itemize}
        \item tournament index \(i\)
        \item tournament size \(N_{i}\)
        \item cooperation rating \((C_{r})\)
        \item rank \((r_{i})\)
        \item normalized rank \(\bar{r_{i}} = \frac{r_{i}}{N_{i}}\)
        \item maximum cooperation rating \(C^{*}_{r}\)
        \item minimum cooperation rating  \(\tilde{C_{r}}\)
        \item normalised rank of maximum cooperator \(\bar{r_{C}}\)
        \item normalised rank of minimum cooperator \(\bar{r_{D}}\)
        \item mean cooperation rating \(\bar{C_{r}}\)
        \item median cooperation rating \(\breve{C_{r}}\)
        \item \(C\) ratio of the winner \(C_{W}\)
        \item \(C\) ratio of the looser \(C_{L}\)
        \item standard deviation of the cooperating ratio \(\sigma\).
    \end{itemize}
    \end{multicols}

\section{Analysis}

\begin{itemize}
\item add \(r\) plots
\item some more plots
\end{itemize}
\end{document}